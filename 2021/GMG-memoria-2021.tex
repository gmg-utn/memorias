\documentclass[a4paper,11pt,twoside,final,titlepage,onecolumn,openright]{report}
\usepackage{fontspec} 
\usepackage[xetex]{geometry} 
\usepackage{xunicode}
\usepackage{xltxtra}
\defaultfontfeatures{Mapping=tex-text}
\setmainfont [Ligatures={Common}]{Linux Libertine O}
\usepackage{microtype} 
\usepackage[spanish]{babel}
\usepackage{csquotes}
\usepackage{graphicx}
\usepackage{amsmath}
\usepackage{textcomp}
\usepackage{fontawesome}
\usepackage{tabularx, booktabs}
\usepackage[lofdepth,lotdepth]{subfig}
\usepackage{url}
\usepackage[sorting=none, url=false, doi=true, eprint=false, isbn=false, maxnames=10]{biblatex}
\addbibresource{gmg-2021.bib}

\usepackage{verbatim}

\usepackage{multicol}
\usepackage{anysize} % Soporte para el comando \marginsize
\marginsize{2.5cm}{2cm}{1.5cm}{1.5cm}
\usepackage{hyperref}
\hypersetup{
  pdfauthor={},
  pdftitle={GMG -- Memoria 2021},
  colorlinks=True,
  linkcolor=blue,
  anchorcolor=black,
  citecolor=blue,
  urlcolor=blue,
  pdftoolbar=false,
}
\renewcommand{\labelitemi}{\faAngleRight}


\begin{document}

\title{Grupo de Materiales Granulares (GMG) \\ Memoria anual para el período 2021 \\ Plan de trabajo 2022}
\date{Abril 2022}

%\maketitle

\chapter*{}

\begin{flushright}
 ISSN 2618-1738
\end{flushright}

\vspace{5cm}
\begin{center}

\textbf{\LARGE Grupo de Materiales Granulares (GMG) \\ Memoria anual para el período 2021 \\ Plan de trabajo 2022} 

\vspace{1cm}
\textbf{\Large Abril 2022}
\end{center}


\chapter*{}

\begin{center}

\textbf{\LARGE UNIVERSIDAD TECNOLÓGICA NACIONAL} 

\vspace{1cm}
\textbf{\Large Rector}

Ing. Ing. Rubén Soro

\vspace{0.5cm}
\textbf{\Large Secretario de Ciencia, Tecnología y Posgrado} 

Ing. Omar Del Gener

\vspace{5cm}

\textbf{\LARGE FACULTAD REGIONAL LA PLATA} 

\vspace{1cm}
\textbf{\Large Decano} 

Mg. Ing. Luis Agustín Ricci

\vspace{0.5cm}
\textbf{\Large Secretario de Ciencia, Tecnología y Posgrado}

Dr. Ing. Gerardo Hugo Botasso
 
\end{center}


\tableofcontents

\chapter{Administración}

El GMG inició sus actividades en el Departamento de Ingeniería Mecánica de la Facultad Regional La Plata en mayo de 2012. Se genera mediante la fusión de un conjunto de investigadores especializados en mecánica estadística de medios granulares del Instituto de Física de Líquidos y Sistemas Biológicos (CONICET-UNLP) con jóvenes investigadores del Dpto. de Ing. Mecánica de la UTN-FRLP a fin de potenciar las capacidades teórico-computacionales y experimentales y a la vez conjugar actividades de investigación básica y aplicada con actividades de transferencia de conocimiento y tecnología. El GMG fue homologado a fines del año 2013 por el Consejo Superior de la Universidad Tecnológica Nacional mediante la resolución 949/2013.
		
\vspace{0.5cm}
{\bf Misión}

\begin{itemize}
 \item Generar conocimiento sobre el comportamiento de materiales granulares y materia activa mediante investigación básica y aplicada.
 \item Llevar adelante desarrollos tecnológicos orientados a mejorar procesos que involucren materiales granulares y materia activa.
 \item Formar recursos humanos con alta calificación en investigación y desarrollo para contribuir al progreso de los sistemas científico, educativo, productivo y administrativo así como de organizaciones gubernamentales y no gubernamentales.
 \item Consolidar un grupo humano comprometido con objetivos comunes de mediano y largo plazo.
\end{itemize}

\vspace{0.5cm}
{\bf Visión}

\begin{itemize}
 \item Convertirnos en un centro de generación de conocimiento y desarrollo tecnológico de vanguardia en el campo de los materiales granulares proveyendo a la industria de herramientas fundamentales para el diseño y optimización de procesos que involucren materiales granulares y materia activa.
 \item Establecernos como un grupo de referencia en el área de los materiales granulares en el ámbito académico con extensiones a temáticas relacionadas en cuanto a lo fenomenológico y a lo instrumental.
\end{itemize}

\vspace{0.5cm}
{\bf Actividades}
\vspace{0.5cm}

El GMG centra sus actividades de investigación y desarrollo en las siguientes áreas

\begin{itemize}
 \item Flujo y atasco de materiales granulares y de materia activa.
 \item Compactación por vibración y cizalla.
 \item Distribución de esfuerzos en materiales granulares y en contenedores.
 \item Estados de la materia granular.
 \item Propiedades disipativas de los medios granulares.
 \item Mezcla y segregación.
 \item Fluencia lenta.
\end{itemize}

Asimismo se ofrecen servicios de transferencia de conocimiento en las siguientes temáticas

\begin{itemize}
 \item Llenado y descarga de silos y tolvas.
 \item Atascamiento en tolvas dosificadoras.
 \item Transporte y deposición de granulados en matrices fluidas.
 \item Amortiguación de vibraciones.
 \item Evacuación de peatones en estado de pánico.
 \item Compactación y fluidización de depósitos.
 \item Diseño de contenedores.
 \item Envejecimiento de depósitos granulares.
 \item Metrología de materiales granulares.
\end{itemize}

El grupo contribuye además a la formación de grado y postgrado en el Dpto. de Ingeniería Mecánica. Sus miembros son docentes en varias cátedras de grado y en cursos de doctorado. Algunos de sus miembros son también docentes de la Universidad Nacional de La Plata.

\vspace{0.5cm}

{\bf Resumen de actividades 2021}

El año 2021 mantuvo en gran parte del año las fuertes restricciones de acceso a la Facultad iniciadas en 2020 como consecuencia de la pandemia provocada por el coronavirus SARS-CoV-2, afectando el desarrollo normal de las actividades académicas. Debido a esto solo fue posible acceder en pocas oportunidades a las instalaciones del grupo durante la primera mitad del año, y solo hacia el final del 2021 se pudieron reiniciar algunas actividades presenciales. Estas limitaciones tuvieron un profundo impacto en el desarrollo de las actividades del grupo, principalmente en lo referente a tareas experimentales.

Sin embargo, gran parte del trabajo de investigación y formación de recursos humanos se realizó en forma remota, tal como se había organizado en el año anterior, por lo que se mantuvieron las líneas de trabajo computacionales, el dictado de cursos y seminarios, y el seguimiento de los becarios de investigación.

En consecuencia, se publicaron ocho trabajos en revistas internacionales con referato (seis en el primer cuartil, uno en el segundo y uno sin asignación de cuartil del ranking Scimago\footnote{\href{https://www.scimagojr.com/}{https://www.scimagojr.com/}}), se participó en cinco congresos nacionales y cuatro internacionales exponiendo un total de 16 trabajos en formato audiovisual y presentaciones orales (en formato virtual).

Se avanzó en el desarrollo de dos tesis doctorales, continuó la ejecución de dos proyectos homologados UTN y se consolidó la cooperación con colegas de otros grupos en el país y en el extranjero.
\vspace{0.5cm}

{\bf Logros más importantes}

Entre los logros más importantes podemos enumerar:

\begin{itemize}
\item Se publicaron ocho trabajos en revistas internacionales con referato, seis de ellos en el primer cuartil del ranking Scimago.
\item Se alcanzó una participación importante en congresos y reuniones científicas.
\item Finalizaron exitosamente dos proyectos financiados por la Agencia Nacional de Promoción Científica y Tecnológica (ANPCyT).
\item Se sostuvieron las principales actividades del grupo (excepto las tareas experimentales) pese a las restricciones establecidas como consecuencia de la pandemia de COVID-19.
\end{itemize}


\section{Individualización del grupo}

\subsection{Nombre y sigla}
 Grupo de Materiales Granulares (GMG)

 \subsection{Sede}
\begin{quote}
Departamento de Ingeniería Mecánica \\
Facultad Regional La Plata\\
Av. 60 Esq. 124\\
1900 La Plata \\
Tel: 0221 - 4124392\\
Email: \href{mailto://granulares@frlp.utn.edu.ar}{granulares@frlp.utn.edu.ar}
\end{quote}


\subsection{Estructura de gobierno}
Director: Carlos Manuel Carlevaro

\subsection{Objetivos y desarrollo}
Los objetivos propuestos en el plan de trabajo para el 2021 fueron alcanzados en gran medida. Se sostuvo el ritmo de publicaciones internacionales con referato superando la cantidad de publicaciones del año 2020. Se logró incrementar significativamente la presentación de trabajos en reuniones científicas. En general, todos los miembros del grupo han participado en la elaboración de trabajos que fueron publicados o presentados en congresos.

Se organizaron 14 seminarios durante el año, coordinados por el Dr. Ariel Meyra, lo que representa un incremento importante respecto del año anterior. Estos fueron dictados por casi todos los miembros del grupo y por expositores externos invitados, enfatizando la presencia de expositoras mujeres en un esfuerzo por visibilizar las dificultades en alcanzar un balance de género en el ámbito de la ingeniería. Al igual que en 2020, la implementación de los seminarios en forma virtual permitió mantener un número alto de asistentes en relación con los años previos a la pandemia.

Un estudiante de grado renovó su beca de investigación, y se incorporaron tres nuevos estudiantes con estas becas. Además participaron de las actividades del GMG tres estudiantes con beca de servicio en el Departamento de Ingenería Mecánica. Los dos estudiantes doctorales del grupo avanzaron en sus investigaciones sin mayores inconvenientes, dado que pudieron desarrollar sus tareas en forma remota. Se realizó con éxito (en forma virtual) el congreso internacional \textit{Powders \& Grains} en el cual el GMG participó en el grupo de organizadores locales. El Dr. Marcos Madrid inició una estancia de investigación en la Universidad Politécnica de Madrid (España) que se extendió por un año. 

Finalmente, concluyeron exitosamente dos proyectos PICT financiado por la ANPCyT (PICT-2016-2658 y  PICT-2016-2303) en los que participaron miembros del GMG, y se elaboraron y presentaron solicitudes de financiamiento a la ANPCyT que fueron otorgadas, y cuyas ejecuciones se iniciarán en 2022. En diciembre de 2021 se inició la actualización completa del sistema operativo y de administración del clúster de cálculo.

En conclusión, los objetivos propuestos en la planificación de actividades del año 2021 se alcanzaron satisfactoriamente.

\section{Personal}

\subsection{Nómina de investigadores}

{\small
\begin{tabular}{l l l c c c}
\toprule
Apellido y nombre & Cargos & Dedicación & Categ. UTN & Incentivos & Horas$^a$ \\
\midrule
Baldini, Mauro           & Prof. Adjunto FRLP     & Simple &   &   & 5 \\
Carlevaro, Carlos Manuel & Prof. Titular FRLP     & Simple & B  & III & 20 \\
                         & Invest. Indep. CONICET    &  &   &\\
Fernández, Matías        & Profesor Adjunto FRLP  & Exclusiva & D & & 45\\
                         & JTP                    & Simple & & & \\
Irastorza, Ramiro Miguel & Prof. Titular FRLP & Simple & C & III   & 15 \\
                             & Invest. Adjunto. CONICET &   &  &  & \\
Madrid, Marcos Andrés    & Invest. Asist. CONICET & Exclusiva & D & III & 45\\
                         & JTP UNLP               & Simple &  & & \\
Meyra, Ariel Germán  & Prof. Adjunto  FRLP    & Simple & D & III & 15 \\
                         & Invest. Adjunto. CONICE T&   &  & &  \\
\bottomrule 
\end{tabular} 
}

\normalsize
\vspace{0.5cm}
$^a$ Sólo se cuenta la dedicación a la investigación sin sumar aquí las horas dedicadas a la docencia o actividades de extensión.


\subsection{Personal profesional}
No se cuenta con este tipo de personal.
% \begin{tabular}{l l l r}
% \toprule
% Apellido y nombre & Cargos & Dedicación & Horas Investig.$^a$\\
% \midrule
% Rosenthal, Gustavo & Ayud. Primera FRLP & Semiexclusiva & 2\\
% \bottomrule
% \end{tabular}

% \normalsize
% \vspace{0.5cm}
% $^a$ Sólo se cuenta la dedicación a la investigación sin sumar aquí las horas dedicadas a la docencia o actividades de extensión.


\subsection{Personal técnico, administrativo y de apoyo}
No se cuenta con este tipo de personal.

% \begin{tabular}{l l r}
% Apellido y nombre & Horas \\
% \hline\hline\\
% Marruedo Eric &  45\\
% \hline \\
% Cagnola Juan Pablo & 45 \\
% \hline\hline 
% \end{tabular}

\subsection{Becarios y personal en formación}

\subsubsection{Tesistas de maestría y/o doctorado}
\begin{tabular}{l l l l r}
\toprule
Apellido y nombre & Tipo de tesis & Inicio & Financ. & Horas$^a$ \\
\midrule
Mosca, Santiago & Doc. Ing. Materiales & 10/2020 &  CONICET & 45\\
Basiuk, Lucas & Doc. Ing. Materiales & 10/2020 &  Sin financiación & 20\\
\bottomrule
\end{tabular}

\normalsize
\vspace{0.5cm}
$^a$ Sólo se cuenta la dedicación a la investigación sin sumar aquí las horas dedicadas a la docencia o actividades de extensión.

\subsubsection{Becarios graduados}

No hubo becarios graduados durante 2021.

% \begin{tabular}{l l l l r}
% \toprule
% Apellido y nombre & Tipo de Beca & Financ. & Horas \\
% \midrule
% Petri, Maximiliano & Beca BINID & UTN & 20\\
% \bottomrule 
% \end{tabular}

\subsubsection{Becarios alumnos}

\begin{tabular}{l l l l r}
\toprule
Apellido y nombre & Tipo de Beca & Financ. & Horas \\
\midrule
Calbucoy, Carla Mariela& Beca SAE FRLP & UTN & 12 \\
Gracia, César & Beca SAE FRLP & UTN & 18 \\
Kjolhede, Erik & Beca SAE FRLP & UTN & 18 \\
Poncetta, Hernán & Beca SAE FRLP & UTN & 18 \\
Cappellari, Gian Franco & Beca de Servicio & UTN & 6 \\
Ferreyra, Marcos & Beca de Servicio & UTN & 6 \\
Alvis, Iver & Beca de Servicio & UTN & 12 \\
\bottomrule 
\end{tabular}

 \subsubsection{Pasantes}
\begin{tabular}{l l r}
\toprule
Apellido y nombre & Financ. & Horas \\
\midrule
Vatalaro, Giancarlo & Sin financiamiento & 5\\
\bottomrule
\end{tabular}

\normalsize
\vspace{0.5cm}

% En virtud de las tareas de investigación llevadas a cabo por los becarios, todos solicitaron ingresar a la Carrera del Investigador UTN, obteniendo luego de la evaluación de sus solicitudes la categoría F (Basiuk, Petri, Recalt y Mosca).

\section{Equipamiento e infraestructura}

\subsection{Equipamiento e infraestructura principal disponible}

El GMG cuenta con dos oficinas, un laboratorio y un cuarto para el cluster de cómputo. Los equipos principales con que se cuenta son

\begin{itemize}
 \item 1 Cluster de cómputo dedicado (248 procesadores con sistema de administración SLURM).
 \item 1 Osciloscopio.
 \item 1 Analizador de redes vectorial.
 \item 2 Placas adquisidoras.
 \item 2 Balanzas electrónicas.
 \item 8 PC de escritorio y para control de dispositivos de laboratorio.
 \item 1 Impresora láser B/N.
 \item Fuente regulada/regulable.
 \item Mobiliario básico de oficina y de laboratorio (escritorios, sillas, mesadas, mesas, armarios, etc.).
 \item Herramientas básicas (llaves, taladro, soldador, multímetro, etc.).
 \item Un banco de prueba para medición de tensiones en silos.
 \item Un sistema robotizado para descarga de silos bidimensionales.
 \item Un banco de prueba para flujo en configuraciones confinadas con bomba peristáltica.
\end{itemize}


\subsection{Locales y aulas}

\begin{itemize}
 \item {\bf Oficina:} Dos oficinas de 22 m$^2$. 
 \item {\bf Cluster:} Cuarto de 4 m$^2$. 
\end{itemize}

\subsection{Laboratorios y talleres}

\begin{itemize}
 \item {\bf Laboratorio A:} Laboratorio de 20 m$^2$.
 \item {\bf Laboratorio B:} Laboratorio de 6 m$^2$.
\end{itemize}

\subsection{Servicios generales}

\begin{itemize}
 \item {\bf Centro de mecanizado:} Servicio prestado por el Dpto. de Ing. Mecánica.
 \item {\bf Talleres:} Servicio prestado por el Dpto. de Ing. Mecánica.
 \item {\bf Biblioteca:} Servicio prestado por la Fac. Regional La Plata, y por la biblioteca propia del Departamento de Ingeniería Mecánica. Adicionalmente se cuenta con el servicio de biblioteca electrónica del Min. de Ciencia, Tecnología e Innovación Productiva. 
\end{itemize}

\subsection{Cambios significativos en el período}
Durante el año 2021 se incorporaron tres servidores de cálculo HP donados por la empresa YPF como parte de su programa de renovación de equipos. Esto permitió sumar un total de 64 nuevos núcleos de cálculo y 48 GB de RAM.

Mediante el financiamiento para Grupos y Centros de UTN, y los PID homologados, se compró una impresora multifunción HP LaserJet137fnw, dos switchs MikroTik Modelo CSS326-24G-2S+RM para la reconfiguración del clúster de cálculo, dos discos de estado sólido Kingston de 960 GB 2.5"{} SATA3 A400 para incrementar la capacidad de almacenamiento del clúster, y un analizador vectorial de redes NanoVNA - H.


\section{Documentación y biblioteca}

El GMG cuenta con una reducida biblioteca que incluye principalmente actas de congresos y libros de resúmenes de eventos científicos en los que han participado sus investigadores, como así también manuales de los instrumentos adquiridos. El material de consulta bibliográfico es mantenido por la biblioteca de la Fac. Regional La Plata, el Departamento de Ingeniería Mecánica y la biblioteca electrónica del Min. de Ciencia Tecnología e Innovación Productiva. 

\chapter{Actividades I+D+i}

\section{Investigaciones}

\subsection{Proyectos en curso}

\begin{itemize}
 \item {\bf PICT ANPCyT:} PICT-2016-2658, 2018-2020, {\bf Atenuación de vibraciones mediante materiales granulares}. Director: Luis Pugnaloni.
 
  {\bf Objetivos:} este proyecto propone realizar una caracterización acabada del comportamiento de amortiguadores granulares (AG) que permita desarrollar herramientas de diseño simples para impulsar su uso en diversas aplicaciones industriales. Los AG son sistemas pasivos usados para atenuar vibraciones en aplicaciones específicas que requieren un bajo mantenimiento o la capacidad de funcionar en ambientes extremos (como altas o bajas temperaturas donde los amortiguadores basados en fluidos viscosos pierden su capacidad). Un amortiguador de partículas consiste en un recinto que contiene material granulado solidario a la estructura vibrante. Las colisiones inelásticas entre los granos y la fricción producen la disipación de energía capaz de atenuar la amplitud de vibración. Sin embargo los AG presentan una respuesta mucho más compleja que un amortiguador viscoso por lo que predecir su comportamiento durante el diseño es actualmente más difícil.

 {\bf Logros:} Durante 2021 se avanzó en el análisis de la respuesta no lineal del amortiguador y en la asa aparente debido al efecto del confinamiento lateral. Esto dio origen a una publicación internacional con referato.
 
 {\bf Dificultades:} Ha habido un retraso en el desarrollo de los experimentos debido al acceso limitado al laboratorio como consecuencia de la pandemia de COVID-19.

\item {\bf PICT ANPCyT:} PICT-2016-2303, 2018-2020, {\bf Desarrollo e implementación de una metodología para la evaluación in vivo de la calidad ósea}. Director: Ramiro Irastorza.
 
{\bf Objetivos:} El objetivo del proyecto es diseñar y construir un método alternativo de evaluación de salud ósea. Para esto se utiliza un enfoque novedoso que son las técnicas de imágenes por microondas.
Al ser éste un método de radiación no ionizante constituye una herramienta más saludable que el Gold Standard Densitometría de rayos X.

 
 {\bf Logros:} Hasta el momento hemos obtenido cinco publicaciones en revistas internacionales relacionadas con el tema. También se formaron recursos humanos, un doctorado y un posdoctorado. Por otro lado, se está construyendo el primer prototipo en colaboración con el Instituto Argentino de Radioastronomía.
 
 {\bf Dificultades:} La mayor complicación fue el retraso de los experimentos previstos por causa de la pandemia.
 
  
\item {\bf PID UTN:} MAUTILP0007746TC, 2020-2023, {\bf Flujo y transporte de material granular en sistemas de interés tecnológico}. Director: Manuel Carlevaro, codirector: Matías Fernández.
 
{\bf Objetivos:} el objetivo general del presente proyecto consiste en contribuir al conocimiento, tanto básico como aplicado, relativo a las características y comportamiento de la materia granular en procesos dinámicos de flujo y transporte, en sistemas de interés en procesos industriales y tecnológicos. Si bien el comportamiento de la materia granular en procesos y dispositivos tecnológicos es muy diverso y complejo, se abordará la descarga de silos en dos y tres dimensiones, así como el transporte de material granular en fracturas angostas.

 
 {\bf Logros:} se estudió el efecto de incorporar fuerzas de repulsión magnética en una de las especies de una mezcla de granos, de diferentes tamaños, en el caudal de descarga de un silo a través de un orificio angosto en dos dimensiones. Se encontró que en un determinado rango de
 valores para el cociente entre los radios, la interacción magnética favorece el flujo aumentando
 el caudal, mientras que en otro rango el efecto es a la inversa. Estos resultados dieron origen a una publicación en una revista internacional con referato y una comunicación a un congreso nacional.
 
 {\bf Dificultades:} no se pudieron realizar los experimentos previstos por la imposibilidad de acceder al laboratorio como consecuencia de la pandemia.
 
\item {\bf PID UTN:} MAUTNLP0006542, 2020-2022, {\bf Propiedades estructurales en carga y descarga de silos}. Director: Marcos Madrid, codirector: Manuel Carlevaro.
 
{\bf Objetivos:} el objetivo general consiste en avanzar sobre la comprensión de los fenómenos que ocurren durante la manipulación de materiales granulares y su aplicación para mejorar tanto los procesos como los diseños y construcción. Como objetivos particulares proponemos: a) predecir la presión en el interior de un silo durante la carga y descarga para diferentes protocolos de llenado; b) predecir las presiones en el interior de un silo reforzado con tensores en diferentes configuraciones durante la carga y descarga del mismo.

 {\bf Logros:}  en el período 2021 se avanzó en la formación de una alumna del último año de la carrera Lic. en Física de la UNLP para realizar su trabajo de tesis de grado en simulaciones tipo DEM para simular diferentes condiciones de carga y descarga de silos.
 
 {\bf Dificultades:} las tareas experimentales programadas debieron ser postergadas como consecuencia de la situación de pandemia.

\end{itemize}

\subsection{Tesis}

Se encuentran en desarrollo dos trabajos de tesis doctoral:
\begin{itemize}
 \item Santiago Mosca: ``Modelización de flujo y transporte en medios porosos''. Director: Manuel Carlevaro. Codirector: Federico Castez (Y-TEC, UNLP).
 \item Lucas Osvaldo Basiuk: ``Diseño computacional de matrices para ingeniería de tejidos optimizadas de manera estocástica''. Director: Manuel Carlevaro. Codirector: Ramiro Irastorza.
\end{itemize}


\subsection{Congresos y reuniones científicas}

{\bf Participación}

\begin{enumerate}
\item \fullcite{apsmm2021a}.
\item \fullcite{apsmm2021b}.
\item \fullcite{pyg2021}.
\item \fullcite{Agro2021}.
\item \fullcite{GMG-042}.
\item \fullcite{trefemac2021a}.
\item \fullcite{trefemac2021b}.
\item \fullcite{trefemac2021c}.
\item \fullcite{rafa2021}
\item \fullcite{rafa2021a}
\item \fullcite{rafa2021b}
\item \fullcite{rafa2021c}
\item \fullcite{cristal2021}
\item \fullcite{sab2021a}
\item \fullcite{sab2021b}
\item \fullcite{fluidos2021gracia}
\end{enumerate}

\vspace{0.25cm}
{\bf Organización}
\vspace{0.5cm}

El GMG participó, junto con otros grupos de la región, en la organización del congreso internacional \textit{Powders \& Grains} que se desarrolló en forma virtual desde el 5 de julio al 6 de agosto de 2021. Manuel Carlevaro participó en la organización de la División ``Mecánica Estadística, Física No-Lineal y Sistemas Complejos'' en la 106º Reunión Anual de la Asociación Física Argentina, que se realizó en forma de webinar en octubre de 2021, mientras que Marcos Madrid lo hizo en la División ``Materia Blanda''.

\subsection{Otras actividades}

\subsubsection{Visitas recibidas y realizadas}
Debido a las restricciones de viajes provocadas por la pandemia de COVID-19, no se recibieron visitantes en el GMG durante el año 2021. El Dr. Marcos Madrid inició una estadía de investigación en la Universidad Politécnica de Madrid, España, que se extendió por un año, para estudiar el efecto de la forma y rugosidad de las paredes en los patrones de descarga de silos. Por su parte, el Dr. Matías Fernández realizó una visita de un mes al Departamento de Ingenería Mecánica y Minera de la Universidad de Jaén, España.

% 
% \subsubsection{Visitantes recibidos}
% 
% \begin{itemize}
%  \item {\bf Luis Pugnaloni:} julio y diciembre, 2019. Investigador de la Universidad Nacional de La Pampa.
%  \item {\bf Lou Kondic:} julio 2019. Investigador del New Jersey Institute of Technology, Estados Unidos.
% \end{itemize}
% 
% \subsubsection{Visitas realizadas}
%  
%  \begin{itemize}
%   \item {\bf Manuel Carlevaro:} febrero 2019. Investigador visitante en el instituto de Química Física Rocasolano, Madrid, España.
%  \end{itemize}

\subsubsection{Otras}

Los miembros del GMG participan además en las siguientes actividades académicas relacionadas con la investigación:

\begin{itemize}
  \item \textbf{Consejo Asesor de Ciencia Tecnología y Postgrado UTN-FRLP}: C. Manuel Carlevaro fue miembro de la comisión durante 2021. 
  
 \item \textbf{Referato de artículos para revistas internacionales}: durante 2021, C. Manuel Carlevaro fue revisor de artículos para las revistas \textit{Journal Of Petroleum Science And Engineering} (Elsevier), \textit{Journal Of Physics Condensed Matter} (IOP Publishing Ltd.), \textit{Scientific Reports} (Nature Publishing Group) y \textit{European Physical Journal E} (Springer New York). R. Irastorza fue revisor de artículos en \textit{Physics in Medicine and Biology} (IOP Publishing Ltd.), \textit{International Journal of Hyperthermia} (Informa Healthcare) y \textit{Biomedical Physics \& Engineering Express} (IOP Publishing Ltd.). Matías Fernández fue revisor de artículos de la revista \textit{Scientific Reports} (Nature Publishing Group) y \textit{Mecánica Tecnológica} (UTN - FRLP).
 
 \item \textbf{Evaluación de personal de CyT}: C. Manuel Carlevaro fue Par Consultor en la Convocatoria Solicitud de Promoción de la Carrera del Investigador 2021 de CONICET y miembro de la Comisión Evaluadora para la Carrera Docente del Área Física de la Facultad de Ciencias Exactas y Naturales de la Universidad Nacional de La Pampa.

\item \textbf{Jurados de tesis doctorales}: C. Manuel Carlevaro fue jurado de una tesis doctoral de la Universidad de Navarra (España) en febrero de 2021 y otra de la Universidad Tecnológica Nacional diciembre de 2021. R. Irastorza fue jurado de una tesis doctoral de la Universidad Tecnológica Nacional en diciembre de 2021.

\item \textbf{Evaluación de proyectos de CyT}: R. Irastorza fue evaluador del programa ``Temas abiertos PICT-2020-SERIEA-I-A Temas Abiertos (I) - Equipo de Trabajo (A)"{}  del FONCYT.
 
\item \textbf{Asociación Física Argentina:} C. Manuel Carlevaro se desempeñó como Vocal Titular en representación de la Filial La Plata en la Comisión Directiva, mientras que M. Madrid forma parte del comité ejecutivo de la división de Materia Blanda.
 
\end{itemize}


\subsection{Trabajos publicados}

\subsubsection{Con referato}

% \bibliographystyle{plain}
% \begingroup
% \makeatletter
% \let\@bibitem\saved@bibitem
% \nobibliography{gmg-2020}
% \endgroup


\begin{enumerate}
    \item \fullcite{madrid2021a}
    \item \fullcite{madrid2021b}.
    \item \fullcite{gonzalez2021}.
    \item \fullcite{irastorza2021}
    \item \fullcite{basak2021}.
    \item \fullcite{vega2021}.
    \item \fullcite{Ferreyra2021}.
    \item \fullcite{fernandez2021}.
\end{enumerate}

\subsubsection{Sin referato}
No se publicaron trabajos sin referato.

\subsubsection{En libros}

No se publicaron trabajos en libros.

\subsubsection{Informes y memorias técnicas}
\begin{enumerate}
 \item Memoria anual del GMG para el período 2020 (ISSN: 2618-1738).
\end{enumerate}


\subsubsection{Tareas de divulgación}

Se realizaron 14 seminarios abiertos, en forma virtual que contaron con la participación de estudiantes, profesores e investigadores:
\begin{itemize}
 \item 18 de mayo: Luis Pugnaloni (Universidad Nacional de La Pampa. ``Variantes de amortiguadores granulares con reducción de efectos no lineales indeseados''.
 \item 1 de junio: Martín Sánchez (YTEC). ``Las discontinuidades mecánicas: su efecto sobre la propagación de fracturas hidráulicas''.
 \item 15 de junio: Ana Cozzarín (Universidad Nacional de La Plata y Universidad Tecnológica Nacional). ``La relevancia del desarrollo de aleaciones estratégicas con ojos de mujer''.
 \item 29 de junio: Paula Gago (Imperial College London). ``Fluctuaciones de densidad en un medio granular atravesando la transición vítrea''.
 \item 13 de julio: Marcelo Martín (Albano-Cosuol). ``Fabricación de autopartes plásticas a través del molde por inyección ''.
 \item 3 de agosto: Lucas Basiuk (GMG). ``Algoritmos genéticos en Ingeniería''.
 \item 17 de agosto: Matías Fernández (GMG). ``Plan de mejoras departamental en apoyo a la investigación''.
 \item 31 de agosto: Alejandra Aguirre (Universidad de Buenos Aires). ``Coeficiente de restitución: el rol de las rotaciones inducidas durante el impacto''.
 \item 14 de septiembre: Victoria Ferreyra (Universidad Nacional de La Pampa). ``Un viaje al corazón de un amortiguador granular''.
 \item 28 de septiembre: Santiago Mosca (GMG). ``Cálculo científico y CAE con Raspberry Pi''.
\item 12 de octubre: María Clelia Mosto (Universidad Nacional de La Plata). ``Ingeniería + Biología = Biología + Ingeniería''.
\item 26 de octubre: Giancarlo Vatalaro (GMG). ``Ingeniería y Jetson Nano: Inferencia''.
\item 9 de noviembre: Erik Kjolhede D'Anunzio (GMG). ``Optimización de ciclos de taladrado''.
\item 23 de noviembre: Manuel Carlevaro (GMG). ``Elementos de teoría de grafos y su aplicación en sistemas granulares''.
\end{itemize}

\section{Registros y patentes}

No se realizaron registros ni patentes.



\chapter{Actividades en docencia}

\section{Docencia de grado}

Los integrantes del GMG son docentes de las siguientes cátedras de la UTN-FRLP.

\begin{itemize}
 \item {\bf Mecánica de materiales granulares:} C. M. Carlevaro
 \item {\bf Mecánica de fluidos:} M. Baldini, S. Mosca.
 \item {\bf Estimulación hidráulica de yacimientos no convencionales:} M. E. Fernández.
 \item {\bf Introducción a los elementos finitos:} R. Irastorza y A. Meyra.
\end{itemize}

Además se participa como docente en otras casas de altos estudios.

\begin{itemize}
 \item {\bf Matemática C (Fac. Ing. UNLP):} M. A. Madrid.
\end{itemize}


\section{Postgrado}

Los docentes del GMG son docentes en los siguientes cursos de postgrado.

\begin{itemize}
 \item {\bf Herramientas computacionales para científicos:} R. Irastorza. A. Meyra y C. M. Carlevaro.
 \item \textbf{Método de elementos finitos con software libre:} R. Irastorza y A. Meyra.
 \item \textbf{Estimulación hidráulica de yacimientos no convencionales:} M. Fernández.
 \item {\bf CAnálisis Estadístico utilizando R (Universidad Nacional Arturo Jauretche):} R. Irastorza.

%  \item {\bf Mecánica y mecánica estadística de materiales granulares:} L. A. Pugnaloni
\end{itemize}

\section{Otras actividades}

Los miembros del GMG dictan seminarios abiertos durante todo el año donde se discuten sus temas de investigación.

El banco de pruebas de descarga de silos montado en los laboratorios del GMG se utiliza para que estudiantes de las cátedras de grado realicen trabajos prácticos experimentales sobre flujo de materiales granulares y distribución de tensiones en un silo. Esta actividad no se ha desarrollado durante 2021 dada la suspensión de clases presenciales debido a la pandemia de COVID-19.

M. Fernández participó en el Tribunal Evaluador del Proyecto Final de Carrera de 23 estudiantes, y fue editor de la revista Mecánica Tecnológica (ISSN: 2683- 9148. Volumen 3, 2021). 

R. Irastorza y A. Meyra participaron en los siguientes siete tribunales de evaluación de Práctica Supervisada (PS):
\begin{itemize}
 \item Córdoba, Agustina. ``Ingeniería de dispositivo para la demostración didáctica del fenómeno de termosifón''.
 \item Novillo, Maximiliano. ``Adecuacióon tecnológica de antorcha 3 - cilp''.
\item Gómez, Martín Atanacio. ``Impulsión Cloacal en la localidad de Marcos Paz''.
\item Ruiz, Ezequiel. ``Verificación estructural del Gancho C''.
\item Curcio, Eibar. ``Planificación de reparaciones, mantenimiento y puesta en marcha de equipos viales''.
\item Odorizzi, Alexis. ``Instalación de calefacción por piso radiante''.
\item Moro, Leandro Manuel. ``Proyecto electrificación rural''.
\end{itemize}



\chapter{Vinculación con el medio socioproductivo}

\section{Transferencia al medio socioproductivo}

Durante el año 2021 se colaboró en el proyecto ``Desarrollo, caracterización y evaluación de agentes de sostén auto-suspendidos para la estimulación de reservorios no convencionales de hidrocarburos'' (PID 2016-0039) cuyo responsable es el Dr. Javier Amalvy del CITEMA, siendo la empresa adoptante YPF Tecnología S.A. (Y-TEC). Se realizaron simulaciones computacionales utilizando el \textit{software} CFDEM para analizar la conductividad en ensayos de agente de sostén sometidos a alta presión.

Se avanzó en el estudio de factibilidad de un proyecto para la optimización de silos con un representante de la fábrica de implementos agrícolas Heedba, de la localidad de 9 de Julio. En consecuencia se generó una propuestas para solicitar financiamiento, que fue aprobada por la ANPCyT (para comenzar en 2022).



\chapter{Informe sobre rendición general de cuentas}

Los valores presentados en la siguiente tabla son estimativos debido a que existen ingresos y erogaciones correspondientes a períodos diferentes del año 2021 dependiendo del inicio y cierre de los subsidios recibidos.

\vspace{1cm}

\footnotesize
\begin{tabular}{ l r r r r r r r r r r }
 \toprule
                   & UTN$^a$   & MAUTILP0007746TC & MAUTNLP0006542 & PICT-2016-2303 & PICT-2016-2658 & Total\\
 \midrule
 \textbf{Ingresos} & 86.250,00 & 59.150,00    & 37.700,00     & 173.917,00  & 0,00   &  357.017,00\\
 \midrule
 \textbf{Erogaciones}\\
    Capital        & 86.250,00      & 59.150,00    & 22.361,00           &       0,00  & 0,00     &       167.761,00 \\
    Corrientes     & 0,00      & 0,00    & 0,00           & 173.917,00  & 0,00     & 173.917,00 \\
 \bottomrule
\end{tabular}

\normalsize
\vspace{0.5cm}
$^a$ Financiamiento de la SCTyP de la UTN para grupos homologados.


\chapter{Programa de actividades 2022}

Las actividades planificadas para el año 2022 son:

\begin{itemize}
\item  Redactar y publicar al menos seis trabajos en revistas internacionales con referato, producto de las investigaciones en las líneas de trabajo actualmente en desarrollo en el GMG.
\item  Participar en al menos tres congresos nacionales y uno internacional. 
\item  Progresar en el desarrollo de los planes de tesis doctorales en curso.
\item  Avanzar en la consolidación de las líneas de trabajo de los investigadores jóvenes. 
\item  Incorporar becarios estudiantes y graduados. 
\item  Continuar y consolidar las colaboraciones existentes con la empresa Y-TEC, el New Jersey Institute of Technology (USA), la Universidad de Navarra y el Instituto de Química Física Rocasolano (España). Manter las colaboraciones activas con el Instituto de Física de Liquidos y Sistemas Biológicos, la Universidad Nacional de La Pampa, la Universidad de Buenos Aires y el Centro Atómico Bariloche.
\item  Dictar cursos de grado y posgrado.
\item  Incorporar a becarios del GMG a la Carrera del Docente Investigador UTN.
\item  Participar en las actividades que proponga la Secretaría de Ciencia y Tecnología de la Facultad Regional La Plata.
\end{itemize}


\end{document}
 
