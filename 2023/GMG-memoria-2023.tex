\documentclass[a4paper,11pt,twoside,final,titlepage,onecolumn,openright]{report}
\usepackage{fontspec} 
\usepackage[xetex]{geometry} 
\usepackage{xunicode}
\usepackage{xltxtra}
\defaultfontfeatures{Mapping=tex-text}
\setmainfont [Ligatures={Common}]{Linux Libertine O}
\usepackage{microtype} 
\usepackage[spanish]{babel}
\usepackage{csquotes}
\usepackage{graphicx}
\usepackage{amsmath}
\usepackage{textcomp}
\usepackage{fontawesome}
\usepackage{tabularx, booktabs}
\usepackage[lofdepth,lotdepth]{subfig}
\usepackage{url}
\usepackage[sorting=none, url=false, doi=true, eprint=false, isbn=false, maxnames=10]{biblatex}
\addbibresource{gmg-2023.bib}

\usepackage{verbatim}

\usepackage{multicol}
\usepackage{anysize} % Soporte para el comando \marginsize
\marginsize{2.5cm}{2cm}{1.5cm}{1.5cm}
\usepackage{hyperref}
\hypersetup{
  pdfauthor={},
  pdftitle={GMG -- Memoria 2023},
  colorlinks=True,
  linkcolor=blue,
  anchorcolor=black,
  citecolor=blue,
  urlcolor=blue,
  pdftoolbar=false,
}
%\renewcommand{\labelitemi}{\faAngleRight}
\renewcommand{\labelitemi}{\faCaretRight}


\begin{document}

\title{\textsc{Grupo de Materiales Granulares (GMG)}\\
Memoria anual para el período 2023 \\
Plan de trabajo 2024}
\date{Abril 2024}

%\maketitle

\chapter*{}

\begin{flushright}
 ISSN 2618-1738
\end{flushright}

\vspace{5cm}
\begin{center}

\textbf{\LARGE \textsc{Grupo de Materiales Granulares (GMG)} \\[1em] \textsc{Memoria anual para el período 2023} \\[1em] \textsc{Plan de trabajo 2024}}

\vspace{1cm}
\textbf{\Large Abril 2024}
\end{center}


\chapter*{}

\begin{center}

\textbf{\LARGE UNIVERSIDAD TECNOLÓGICA NACIONAL} 

\vspace{1cm}
\textbf{\Large Rector}

Ing. Rubén Soro

\vspace{0.5cm}
\textbf{\Large Secretario de Ciencia, Tecnología y Posgrado} 

Ing. Omar Del Gener

\vspace{5cm}

\textbf{\LARGE FACULTAD REGIONAL LA PLATA} 

\vspace{1cm}
\textbf{\Large Decano} 

Mg. Ing. Luis Agustín Ricci

\vspace{0.5cm}
\textbf{\Large Secretario de Ciencia, Tecnología y Posgrado}

Dr. Ing. Gerardo Hugo Botasso
 
\end{center}


\tableofcontents

\chapter{Administración}

El GMG inició sus actividades en el Departamento de Ingeniería Mecánica de la Facultad Regional La Plata en mayo de 2012. Se genera mediante la fusión de un conjunto de investigadores especializados en mecánica estadística de medios granulares del Instituto de Física de Líquidos y Sistemas Biológicos (CONICET-UNLP) con jóvenes investigadores del Departamento de Ing. Mecánica de la UTN-FRLP a fin de potenciar las capacidades teórico-computacionales y experimentales y a la vez conjugar actividades de investigación básica y aplicada con actividades de transferencia de conocimiento y tecnología. El GMG fue homologado a fines del año 2013 por el Consejo Superior de la Universidad Tecnológica Nacional mediante la resolución 949/2013.
		
\vspace{0.5cm}
{\bf Misión}

\begin{itemize}
 \item Generar conocimiento sobre el comportamiento de materiales granulares y materia activa mediante investigación básica y aplicada.
 \item Llevar adelante desarrollos tecnológicos orientados a mejorar procesos que involucren materiales granulares y materia activa.
 \item Formar recursos humanos con alta calificación en investigación y desarrollo para contribuir al progreso de los sistemas científico, educativo, productivo y administrativo así como de organizaciones gubernamentales y no gubernamentales.
 \item Consolidar un grupo humano comprometido con objetivos comunes de mediano y largo plazo.
\end{itemize}

\vspace{0.5cm}
{\bf Visión}

\begin{itemize}
 \item Convertirnos en un centro de generación de conocimiento y desarrollo tecnológico de vanguardia en el campo de los materiales granulares proveyendo a la industria de herramientas fundamentales para el diseño y optimización de procesos que involucren materiales granulares y materia activa.
 \item Establecernos como un grupo de referencia en el área de los materiales granulares en el ámbito académico con extensiones a temáticas relacionadas en cuanto a lo fenomenológico y a lo instrumental.
\end{itemize}

\vspace{0.5cm}
{\bf Actividades}
\vspace{0.5cm}

El GMG centra sus actividades de investigación y desarrollo en las siguientes áreas

\begin{itemize}
 \item Flujo y atasco de materiales granulares y de materia activa.
 \item Compactación por vibración y cizalla.
 \item Distribución de esfuerzos en materiales granulares y en contenedores.
 \item Estados de la materia granular.
 \item Propiedades disipativas de los medios granulares.
 \item Mezcla y segregación.
 \item Fluencia lenta.
\end{itemize}

Asimismo se ofrecen servicios de transferencia de conocimiento en las siguientes temáticas

\begin{itemize}
 \item Llenado y descarga de silos y tolvas.
 \item Atascamiento en tolvas dosificadoras.
 \item Transporte y deposición de granulados en matrices fluidas.
 \item Amortiguación de vibraciones.
 \item Evacuación de peatones en estado de pánico.
 \item Compactación y fluidización de depósitos.
 \item Diseño de contenedores.
 \item Envejecimiento de depósitos granulares.
 \item Metrología de materiales granulares.
\end{itemize}

El grupo contribuye además a la formación de grado y postgrado en el Departamento de Ingeniería Mecánica. Sus miembros son docentes en varias cátedras de grado y en cursos de doctorado. Algunos de sus miembros son también docentes de la Universidad Nacional de La Plata.

\vspace{0.5cm}

{\bf Resumen de actividades 2023}

Durante el año 2023 se desarrollaron normalmente las tareas del grupo, cumpliendo holgadamente los objetivos planteados en el ``Programa de actividades 2023'' propuesto en el documento ``\textit{Memoria Anual para el Período 2022 - Plan de Trabajo 2023}''. 

Se sostuvo la producción científica del grupo, alcanzando un número considerable de publicaciones internacionales con referato (por encima del promedio de los cinco años previos), y se alcanzó una participación significativa en congresos nacionales e internacionales. Se finalizaron exitosamente dos proyectos homologados UTN y se presentaron propuestas para nuevos PID. Se mantuvieron las colaboraciones con grupos e investigadores nacionales y del exterior.

Se continuó con la formación de recursos humanos de grado y posgrado, a través del dictado de asignaturas de grado, cursos de posgrado, y en la dirección de becarios y tesistas, destacándose la incorporación de un nuevo tesista doctoral con una beca del CONICET, y continuado el desarrollo de dos tesis doctorales iniciadas en los años previos.

\vspace{0.5cm}

{\bf Logros más importantes}

Entre los logros más importantes podemos enumerar:

\begin{itemize}
    \item Se publicaron nueve trabajos en revistas internacionales con referato, seis de ellos en el primer cuartil del ranking Scimago\footnote{\url{https://www.scimagojr.com/}} y tres en el segundo.
\item Se alcanzó una participación importante en congresos y reuniones científicas.
\item Se desarrollaron en nuestra facultad tres cursos de posgrado con una alta participación de estudiantes.
\item Se organizaron dos jornadas científico-técnicas.
\item Se incorporó un nuevo tesista doctoral con beca de CONICET.
\end{itemize}


\section{Individualización del grupo}

\subsection{Nombre y sigla}
 Grupo de Materiales Granulares (GMG)

 \subsection{Sede}
\begin{quote}
Departamento de Ingeniería Mecánica \\
Facultad Regional La Plata\\
Av. 60 Esq. 124\\
1900 La Plata \\
Tel: 0221 - 4124392\\
Email: \href{mailto://granulares@frlp.utn.edu.ar}{granulares@frlp.utn.edu.ar}
\end{quote}


\subsection{Estructura de gobierno}
Director: Carlos Manuel Carlevaro

\subsection{Objetivos y desarrollo}
Los objetivos propuestos en el plan de trabajo para el 2022 fueron alcanzados en gran medida. Se sostuvo el ritmo de publicaciones internacionales con referato superando la cantidad de publicaciones del año 2021. La presentación de trabajos en reuniones científicas fue menor que la del período anterior, no obstante alcanzó un número importante. En general, todos los miembros del grupo han participado en la elaboración de trabajos que fueron publicados o presentados en congresos.

Se organizaron cinco seminarios abiertos durante el año, coordinados por el Dr. Ariel Meyra, lo que representa una disminución importante respecto del año anterior, como consecuencia de la organización del espacio ``Laboratorios abiertos'' organizado por la Secretaría de Ciencia y Tecnología de la Facultad, en el que los miembros del GMG participaron asistiendo y abriendo este ciclo. Los seminarios del GMG fueron dictados, en esta oportunidad, por expositores externos invitados. A diferencia del 2021, la implementación de los seminarios se realizó en forma híbrida, siendo presencial en la Facultad pero transmitido en forma de \textit{streaming} a través de la plataforma Zoom, lo que permitió la asistencia en forma remota de alumnos y expositores. Se organizaron dos jornadas en nuestra Facultad (un \textit{workshop} sobre materiales granulares y un \textit{PyDay} - Día Python). 

Dos estudiantes de grado renovaron su beca de investigación, se incorporaron cuatro nuevos estudiantes (dos con becas Manuel Belgrano), y un graduado participó mediante una beca BINID. Uno de los becarios estudiantes se postuló y obtuvo una beca doctoral de CONICET, a desarrollarse en 2023. Los dos estudiantes doctorales del grupo avanzaron en sus investigaciones sin mayores inconvenientes, asistieron y aprobaron dos cursos de posgrado ( ``\textit{Physics and applications of granular matter}'' y ``Los laberintos del conocimiento científico, teorías, metodologías''), y participaron en la ``Jornada de Doctorandos en Ingeniería UTN 2022''.  El Dr. Marcos Madrid finalizó una estancia de investigación en la Universidad Politécnica de Madrid (España) mientras que el Dr. Ramiro Irastorza estuvo tres meses en el grupo Bio-MIT del Departamento de Ingenería Electrónica de la Universitat Politècnica de València (UPV), España. El Ing. Santiago Mosca realizó una estancia de entrenamiento de dos meses en el Institut de Recherche de Chimie Paris, dependiente de la École Nationale Supérieure de Chimie de Paris (Francia).

Finalmente, concluyó exitosamente un proyecto PID, se presentaron a evaluación tres nuevos PID,  y se iniciaron dos nuevos proyecto financiados por la ANPCyT. En los primeros meses de 2022 se completó la actualización del sistema operativo y de administración del clúster de cálculo.

En conclusión, los objetivos propuestos en la planificación de actividades del año 2022 se alcanzaron satisfactoriamente.

\section{Personal}

\subsection{Nómina de investigadores}

{\small
\begin{tabular}{l l l c c c}
\toprule
Apellido y nombre & Cargos & Dedicación & Categ. UTN & Incentivos & Horas$^a$ \\
\midrule
Carlevaro, Carlos Manuel & Prof. Titular FRLP     & Simple & B  & III & 20 \\
                         & Invest. Indep. CONICET    &  &   &\\
Fernández, Matías        & Profesor Adjunto FRLP  & Exclusiva & D & & 30\\
                         & JTP                    & Simple & & & \\
Irastorza, Ramiro Miguel & Prof. Titular FRLP & Simple & B & III   & 15 \\
                             & Invest. Adjunto. CONICET &   &  &  & \\
Madrid, Marcos Andrés    & Invest. Asist. CONICET & & C & III & 20\\
                         & JTP UNLP               & &  & & \\
Meyra, Ariel Germán  & Prof. Adjunto  FRLP    & Simple & C & III & 15 \\
                         & Invest. Adjunto. CONICET &   &  & &  \\
\bottomrule 
\end{tabular} 
}

\normalsize
\vspace{0.5cm}
$^a$ Sólo se cuenta la dedicación a la investigación sin sumar aquí las horas dedicadas a la docencia o actividades de extensión.


\subsection{Personal profesional}
No se cuenta con este tipo de personal.
% \begin{tabular}{l l l r}
% \toprule
% Apellido y nombre & Cargos & Dedicación & Horas Investig.$^a$\\
% \midrule
% Rosenthal, Gustavo & Ayud. Primera FRLP & Semiexclusiva & 2\\
% \bottomrule
% \end{tabular}

% \normalsize
% \vspace{0.5cm}
% $^a$ Sólo se cuenta la dedicación a la investigación sin sumar aquí las horas dedicadas a la docencia o actividades de extensión.


\subsection{Personal técnico, administrativo y de apoyo}

\begin{tabular}{l l r}
\toprule
Apellido y nombre & Horas \\
\midrule
 Villalba, Manuel &  40\\
 \bottomrule
 \end{tabular}

\subsection{Becarios y personal en formación}

\subsubsection{Tesistas de maestría y/o doctorado}
\begin{tabular}{l l l l r}
\toprule
Apellido y nombre & Tipo de tesis & Inicio & Financ. & Horas$^a$ \\
\midrule
Basiuk, Lucas & Doc. Ing. Materiales & 10/2020 &  CONICET & 40 \\
Gracia, César & Doc. Ing. Materiales & 4/2023 &  CONICET & 40 \\
Mosca, Santiago & Doc. Ing. Materiales & 10/2020 &  CONICET & 40\\
\bottomrule
\end{tabular}

\normalsize
\vspace{0.5cm}
$^a$ Sólo se cuenta la dedicación a la investigación sin sumar aquí las horas dedicadas a la docencia o actividades de extensión.

\subsubsection{Becarios graduados}

 \begin{tabular}{l l l l r}
 \toprule
 Apellido y nombre & Tipo de Beca & Financ. & Horas \\
 \midrule
 El Ahmar, Elías & Beca BINID & UTN & 20\\
 \bottomrule 
 \end{tabular}

\subsubsection{Becarios alumnos}

\begin{tabular}{l l l l r}
\toprule
Apellido y nombre & Tipo de Beca & Financ. & Horas \\
\midrule
Brik, Martín & Beca SCyT & UTN & 12 \\
Torres, Miguel & Beca SCyT & UTN & 12 \\
Zonco, Sofía & Beca SCyT & UTN & 12 \\
\bottomrule 
\end{tabular}

 \subsubsection{Pasantes}
\begin{tabular}{l l r}
\toprule
Apellido y nombre & Financ. & Horas \\
\midrule
Vatalaro, Giancarlo & Sin financiamiento & 5\\
\bottomrule
\end{tabular}

\normalsize
\vspace{0.5cm}

% En virtud de las tareas de investigación llevadas a cabo por los becarios, todos solicitaron ingresar a la Carrera del Investigador UTN, obteniendo luego de la evaluación de sus solicitudes la categoría F (Basiuk, Petri, Recalt y Mosca).

\section{Equipamiento e infraestructura}

\subsection{Equipamiento e infraestructura principal disponible}

El GMG cuenta con dos oficinas, un laboratorio y un cuarto para el cluster de cómputo. Los equipos principales con que se cuenta son

\begin{itemize}
 \item 1 Cluster de cómputo dedicado (248 procesadores con sistema de administración SLURM).
 \item 1 Osciloscopio.
 \item 1 Analizador de redes vectorial.
 \item 2 Placas adquisidoras.
 \item 2 Balanzas electrónicas.
 \item 8 PC de escritorio y para control de dispositivos de laboratorio.
 \item 1 Impresora láser B/N.
 \item Fuente regulada/regulable.
 \item Mobiliario básico de oficina y de laboratorio (escritorios, sillas, mesadas, mesas, armarios, etc.).
 \item Herramientas básicas (llaves, taladro, soldador, multímetro, etc.).
 \item Un banco de prueba para medición de tensiones en silos.
 \item Un sistema robotizado para descarga de silos bidimensionales.
 \item Un banco de prueba para flujo en configuraciones confinadas con bomba peristáltica.
 \item Un agitador de varilla con conjunto soporte.
 \item Un multímetro con termocupla tipo J.
\item Impresora 3D.
\item Dos cilindros de acrílico para estudio de silos.
\item Dos notebooks ASUS con procesador Ryzen-7.
\item Compresor de aire.
\end{itemize}


\subsection{Locales y aulas}

\begin{itemize}
 \item {\bf Oficina:} Dos oficinas de 22 m$^2$. 
 \item {\bf Cluster:} Cuarto de 4 m$^2$. 
\end{itemize}

\subsection{Laboratorios y talleres}

\begin{itemize}
 \item {\bf Laboratorio A:} Laboratorio de 20 m$^2$.
 \item {\bf Laboratorio B:} Laboratorio de 6 m$^2$.
\end{itemize}

\subsection{Servicios generales}

\begin{itemize}
 \item {\bf Centro de mecanizado:} Servicio prestado por el Departamento de Ing. Mecánica.
 \item {\bf Talleres:} Servicio prestado por el Departamento de Ing. Mecánica.
 \item {\bf Biblioteca:} Servicio prestado por la Facultad Regional La Plata, y por la biblioteca propia del Departamento de Ingeniería Mecánica. Adicionalmente se cuenta con el servicio de biblioteca electrónica del Min. de Ciencia, Tecnología e Innovación Productiva. 
\end{itemize}

\subsection{Cambios significativos en el período}
Durante el año 2023 se incorporó un multímetro con termocupla tipo J, una impresora 3D, dos cilindros de acrílico para el estudio de silos, dos notebook para el control y registro de experimentos, y un compresor de aire.

\section{Documentación y biblioteca}

El GMG cuenta con una reducida biblioteca que incluye principalmente actas de congresos y libros de resúmenes de eventos científicos en los que han participado sus investigadores, como así también manuales de los instrumentos adquiridos. El material de consulta bibliográfico es mantenido por la biblioteca de la Facultad Regional La Plata, el Departamento de Ingeniería Mecánica y la biblioteca electrónica del Ministerio de Ciencia Tecnología e Innovación Productiva. 

\chapter{Actividades I+D+i}

\section{Investigaciones}

\subsection{Proyectos en curso}

\begin{itemize}
  
\item {\bf PID UTN:} MAUTILP0007746TC, 2020-2023, {\bf Flujo y transporte de material granular en sistemas de interés tecnológico}. Director: Manuel Carlevaro, codirector: Matías Fernández.
 
{\bf Objetivos:} el objetivo general del presente proyecto consiste en contribuir al conocimiento, tanto básico como aplicado, relativo a las características y comportamiento de la materia granular en procesos dinámicos de flujo y transporte, en sistemas de interés en procesos industriales y tecnológicos. Si bien el comportamiento de la materia granular en procesos y dispositivos tecnológicos es muy diverso y complejo, se abordará la descarga de silos en dos y tres dimensiones, así como el transporte de material granular en fracturas angostas.

 
{\bf Logros:} se finalizó exitosamente el proyecto alcanzando los objetivos propuestos. Durante 2023, se logró caracterizar la evolución de las cadenas de fuerza durante el movimiento de \textit{stick-slip} de un cuerpo en el seno de material granular, por medio de medidas topológicas obtenidas a partir de mediciones experimentales. 

 {\bf Dificultades:} no se produjeron dificultades en el desarrollo del proyecto. 
 
\item {\bf PID UTN:} MAUTNLP0009875, 2023-2025, {\bf Propiedades estructurales en carga y descarga de silos}. Director: Marcos Madrid, codirector: Manuel Carlevaro.
 
{\bf Objetivos:} el objetivo general consiste en avanzar sobre la comprensión de los fenómenos que ocurren durante la manipulación de materiales granulares y su aplicación para mejorar tanto los procesos como los diseños y construcción. Como objetivos particulares proponemos: a) predecir la presión en el interior de un silo durante la carga y descarga para diferentes protocolos de llenado; b) predecir las presiones en el interior de un silo reforzado con tensores en diferentes configuraciones durante la carga y descarga del mismo.

 {\bf Logros:} En el período 2023 se avanzó en la formación de una alumna del último año de la carrera de Licenciatura en Física de la Universidad Nacional de La Plata para realizar su trabajo de tesis de grado en las tareas experimentales sobre los silos a escala del GMG. 
 
 {\bf Dificultades:} no se produjeron dificultades en el desarrollo del proyecto. 

\item \textbf{PICT ANPCyT:} PICT2020-SERIEA-00457, 2022-2024, {\bf Tomografía de microondas: algoritmos de reconstrucción, validación experimental y aplicaciones}. Director: Ramiro Irastorza.

    \textbf{Objetivos:} La técnica de tomografía por microondas ha despertado gran interés durante la última década ya que permite obtener imágenes con fines médicos de manera no invasiva y con costos muy bajos comparados con otras técnicas de imágenes, por ejemplo, la resonancia magnética. En este proyecto se propone desarrollar métodos de reconstrucción de imágenes por microondas sobre un prototipo experimental desarrollado recientemente en IFLySiB-IAR con la finalidad de evaluar tejidos in-vivo. El presente plan contempla experimentos en laboratorio y la implementación de algoritmos clásicos de reconstrucción como así también el desarrollo de nuevos métodos basados en la teoría del sensado comprimido y la utilización de arquitecturas de inteligencia artificial conocidas como Deep-Learning. El desarrollo de este proyecto permitirá la consolidación del equipo de trabajo y la adquisición de know-how sobre esta nueva tecnología con perspectivas de aplicación a diversos campos de la salud (salud ósea, cáncer de mama, accidentes cerebrovasculares, etc.) como así también en aplicaciones industriales, por ejemplo, en la industria agroalimentaria (monitoreo de silos, determinación de nivel humedad de granos en línea, búsqueda de fallas y determinación de calidad de maderas, etc.).

    \textbf{Logros:}  se ha avanzado en el desarrollo del software para la reconstrucción de imágenes en microondas (\url{https://github.com/rirastorza/Intro2MI}). Con este repositorio se busca, no solo resolver el problema de reconstrucción tomográfica, sino también formar a los potenciales estudiantes y becarias/os en la simulación de problemas electromagnéticos en microondas. También estamos trabajando en la construcción de una base de datos (\url{https://github.com/rirastorza/heelSimulationDB}) para simulaciones de cortes de muñeca y tobillo, y en el desarrollo de técnicas que involucran inteligencia artificial en la resolución del problema inverso (\url{https://doi.org/10.48550/arXiv.2308.03818}). 

\textbf{Dificultades:} aunque el sistema experimental está implementado y se corrigieron los problemas de motores paso a paso y engranajes, la caracterización del tomógrafo tanto en laboratorio común como en cámara anecoica, presentan diferencias sustanciales con las simulaciones y algunos resultados que no hemos logrado interpretar correctamente, continuamos trabajando en ello. 

\item \textbf{PICT ANPCyT:} PICT-2020-SERIEA-I-GRF, 2022-2024, Análisis de propiedades tensionales en carga y descarga de silos. Director: Marcos Madrid, codirector: Ariel Meyra.

    \textbf{Objetivos:} El objetivo principal del presente proyecto es avanzar en la comprensión de los fenómenos físicos que ocurren durante la manipulación (carga, descarga, almacenamiento) de materiales granulares. 

    \textbf{Logros:} En el periodo 2023 se avanzó en la construcción de dos nuevos silos prototipo y se realizaron series de experimentos con diferentes protocolos de llenado y diferentes obstáculos en su interior y se realizaron diferentes simulaciones y experimentos de silos con paredes rugosas.

    \textbf{Dificultades:} No se registraron dificultades durante este período.

\item \textbf{PIP CONICET:} PIP 2021-2023 GI, 11220200100717CO, 2021-2023, Descarga forzada de materiales granulares. Director: Luis Pugnaloni, Co-Director: Marcos Madrid.

    \textbf{Objetivos:} El objetivo general es mejorar nuestro conocimiento básico del comportamiento de materiales granulares. Esto nos permitirá eventualmente ayudar a mejorar procesos industriales que requieren el almacenamiento y manipulación de materiales granulares.

    \textbf{Logros:} Se licenció en física un alumno de grado en la UNLPam. Se realizaron experimentos y simulaciones de descargas de silos con diferentes pistones (descargas forzadas). Se continuó trabajando en el desarrollo de un modelo teórico que permita predecir el caudal de granos a través de un orificio con y sin pistón.

    \textbf{Dificultades:} No hubo dificultades en la ejecución del proyecto durante 2022.

\item {\bf PID UTN:} MAECLP0009851TC, 2023-2026, {\bf Resolución de problemas Biomédicos y Biomiméticos por Elementos Finitos}. Director: Ramiro M. Irastorza.
 
{\bf Objetivos:} El objetivo general de la investigación es desarrollar y poner a punto la metodología para la resolución de problemas mediante el Método de Elementos Finitos, principalmente relacionados con la Ingeniería Mecánica y con geometrías complejas y de gran cantidad de elementos. Se trabajará construyendo geometrías a partir de reconstrucciones tomográficas, lo cual constituye un importante desafío dado que generalmente los archivos STL provenientes de los programas de recontrucción no son los adecuados para la simulación por elementos finitos. Se atacarán problemas con el Método de Elementos Finitos en dos aplicaciones: (i) ablación por medio de energía electromagnética y (ii) biomecánica. Respecto del tema (i): 1. Desarrollar modelos de torso completo para la simulación de ablación por radiofrecuencia aplicado a arritmias cardíacas. 2. Se propone estudiar por simulación la técnica de campo eléctrico pulsado (PEF). En esta técnica se busca el efecto de electroporación, y el daño provocado no es térmico, si no eléctrico. Objetivos particulares del tema (ii): 3. Relacionados con el procedimiento de discoplastía, el objetivo es construir geometrías de pacientes “tipo” a partir de tomografías o resonancias magnéticas y evaluar de manera cuantitativa distancias, superficies, volúmenes, y ángulos de los dominios a simular. Se espera concluir y obtener mediciones más controladas que las tomadas en dos dimensiones por los médicos. 4. Continuando con el objetivo anterior, se propone simular el par de vértebras lumbares L4 y L5 (las más afectadas en estos casos) y estudiar los modelos mecánicos que se aplican a estos tejidos. Al poseer imágenes pre y pos operación (con implante de PMMA), es posible realizar simulaciones en las dos condiciones y evaluar las diferencias en tensiones, deformaciones y desplazamientos. Finalmente, un objetivo importante en este proyecto es consolidar un grupo multidisciplinario de investigación dedicado a la simulación utilizando Métodos de Elementos Finitos en general, y en particular en aplicaciones biomédicas, biológicas y biomiméticas. 

{\bf Logros:} se avanzó en el modelo de torso completo inspirado en geometrías humanas donde se demostró que la influencia de la ubicación del electrodo pasivo no tiene significancia clínica si se observa el tamaño de lesión producida en el tratamiento de arritmia cardíaca, se publicaron dos trabajos en revistas internacionales. Adicionalmente, en la temática de biomecánica, se demostró que las herramientas de segmentación, análisis de las imágenes, y finalmente modelado con el Método de Elementos Finitos permiten cuantificar objetivamente cada una de las intervenciones y, potencialmente, podrían ser de utilidad en la planificación de las estrategias para la cirugía y evaluación de los resultados postoperatorios.

 {\bf Dificultades:} no se produjeron dificultades en el desarrollo del proyecto.

\item \textbf{PICT ANPCyT:} PICT-2021-I-A-00294, 2023 -- 2026, Experiments and modeling of particle dampers with obstacles. Director: Luis Pugnaloni. Integrante del grupo responsable: Manuel Carlevaro.

    \textbf{Logros: } Se inició el desarrollo de un \textit{software} para la modelización y simulación en dos dimensiones de un amortiguador granular cuya geometría se optimiza por medio de un algoritmo genético. Se realizaron pruebas de concepto en laboratorio para la construcción de un banco de ensayos de amortiguadores granulares.

\textbf{Dificultades:} no se presentaron dificultades durante el primer año del proyecto.

\item \textbf{FITBA:} Primera Convocatoria del ``Fondo de Innovación Tecnológica de Buenos Aires'', proyecto A64, Optimización del consumo de energía en sistemas de aireación de silos. Director: C. Manuel Carlevaro.

    \textbf{Logros: } Se realizó la selección y adquisición de sensores de temperatura y humedad, los cuales fueron caracterizados y calibrados. Se construyeron dos silos a escala que incoporaron los sistemas de ventilación a evaluar, junto con el sistema de adquisición de datos. Se realizaron las mediciones de laboratorio. 

\textbf{Dificultades:} Se presentaron dificultades en la compra de material y equipamiento, debido a los procesos administrativos de la Universidad. 

\end{itemize}

\subsection{Tesis}

Se encuentran en desarrollo tres trabajos de tesis doctoral:
\begin{itemize}
 \item Santiago Mosca: ``Modelización de flujo y transporte en medios porosos''. Director: Manuel Carlevaro. Codirector: Federico Castez (Y-TEC, UNLP).
 \item Lucas Osvaldo Basiuk: ``Diseño computacional de matrices para ingeniería de tejidos optimizadas de manera estocástica''. Director: Manuel Carlevaro. Codirector: Ramiro Irastorza.
\item César Gracia: ``Incidencia de las características reológicas de un fluido de fractura en el transporte y sedimentación del agente de sostén en estimulación hidráulica de yacimientos''. Director: Manuel Carlevaro. Codirector: Matías Fernández.
\end{itemize}

Durante el año 2023, César Gracia obtuvo una Beca Doctoral de CONICET.

\subsection{Trabajos publicados}

\subsubsection{Con referato}

\begin{enumerate}
    \item \fullcite{luque2023}.
    \item \fullcite{barbosa2023}.
    \item \fullcite{alvarez2023}.
    \item \fullcite{cervantes2023}.
    \item \fullcite{basak2023}.
    \item \fullcite{gago2023}.
    \item \fullcite{irastorza2023}.
    \item \fullcite{ciach2023}.
    \item \fullcite{serna2023}.
\end{enumerate}

\subsubsection{Sin referato}
No se publicaron trabajos sin referato.

\subsection{Congresos y reuniones científicas}

\subsubsection{Internacionales con referato}
\begin{enumerate}
    \item \fullcite{Pinilla2023}.
    \item \fullcite{collavini2023}.
    \item \fullcite{Partec2023}.
\end{enumerate}

\subsubsection{Nacionales}

\begin{enumerate}
    \item \fullcite{cosapro2023}.
    \item \fullcite{trefemac2023a}.
    \item \fullcite{trefemac2023b}.
    \item \fullcite{trefemac2023c}.
    \item \fullcite{trefemac2023d}.
    \item \fullcite{trefemac2023f}.
    \item \fullcite{trefemac2023e}.
    \item \fullcite{rafa2023}.
    \item \fullcite{rafa2023b}.
    \item \fullcite{expoYTEC2023}.
    \item \fullcite{basiuk2023}.
    \item \fullcite{CCUS2023}.
    \item \fullcite{sab2023a}.
    \item \fullcite{sab2023b}.
\end{enumerate}

\subsubsection{En libros y actas de congresos}

No se realizaron publiciones en libros u actas de congresos durante este período.
%\begin{enumerate}
    %\item \fullcite{basiuk2022b}.
%\end{enumerate}

\subsubsection{Informes y memorias técnicas}
\begin{enumerate}
 \item Memoria anual del GMG para el período 2022 (ISSN: 2618-1738).
\end{enumerate}
\vspace{0.25cm}
\section{Organización}
\vspace{0.5cm}

El GMG organizó el ``Segundo \textit{workshop} regional de materiales granulares'', el día 9 de mayo, como actividad satélite del XX Congreso Regional de Física Estadística y Aplicaciones a la Materia Condensada (TREFEMAC), en el CTDR ``Los Reyunos'' de la ciudad de San Rafael, Mendoza. En el encuentro participaron investigadores y estudiantes de la Universidad Nacional de La Pampam, de la Universidad Nacional de San Luis y de la Universidad de Mendoza, además de un investigador extranjero (Universidad de Navarra, España). Se expusieron los siguientes temas:
\begin{itemize}
 \item Caracterización del flujo de un silo modelo con un obstáculo móvil (Anuar Yamil Sirur Flores*, Rodolfo Uñac, Ana María Vidales, Jesica Benito) 
 
 \item Resuspensión de partículas micrométricas (Jesica Benito*, Camila Villagrán Olivares, Rodolfo Uñac, Ana María Videles)
 
 \item Perfiles de presión en el fondo de un silo durante descargas forzadas, María Victoria Ferreyra*, Luis A. Pugnaloni y Diego Maza
 
 \item Estudio de los efectos no lineales en amortiguadores granulares con obstáculos, Julián Gómez Paccapelo*, Ramiro Suarez, María Victoria Ferreyra y Luis A. Pugnaloni.
 
 \item Mejora del flujo de descarga granular por orificios pequeños mediante el agregado de micro-partículas. Sandip H. Gharat y Luis A. Pugnaloni*
 
 \item Desarrollo de un equipo automatizado de sujetador granular: desafíos y soluciones en el diseño. Tomás Navarro-Febre*,  F. Agustín Lehr, F. Valentín Miguez Mareque, Luis A. Pugnaloni.
 
\item Estudio del flujo de granos vibrados a través de orificios mediante la adición de partículas pequeñas en silo bidimensional. Montero Julián*, Gazzano D. Gabriel, Pugnaloni A. Luis.

\item Simulaciones de sistemas granulares con aplicaciones en astrofisica. E. Bringa*, B. Planes, E. Millan, G. Parisi y H. Urbassek.

\item Sobre el rol de la forma de las partículas en el flujo de descarga de silos. D. Maza*, H. Hanif, D. Van der Meer.

\item Efecto del protocolo de llenado sobre la presión en carga y descarga de silos. Deborah Gonzalez y Marcos Madrid*.
\end{itemize}

{\small *: autor expositor.}

El grupo también organizó el ``PyDay La Plata 2023'', con el apoyo de la Asociación Civil Python Argentina\footnote{\url{https://ac.python.org.ar/}}. Este evento se realizó el día 9 de septiembre en el Salón de Actos Presidente Juan Domingo Perón de la Facultad Regional La Plata, y contó con la presencia de más de un centenar de asistentes que participaron en las siete charlas programadas\footnote{Se puede ver el cronograma en \url{https://eventos.python.org.ar/events/pyday-laplata-2023/}.}.


\subsection{Otras actividades}

\subsubsection{Visitas recibidas y realizadas}

Durante el año 2023 se recibió la visita de Ryan Kozlowski, del \textit{College of The Holy Cross} (Worcester, Estados Unidos), con quien se mantiene una extensa colaboración en el estudio del fenómeno de \textit{stick-slip} y más recientemente, en la descarga de silos con granos bidispersos. También, en el marco del proyecto DISCO2Store, visitaron el grupo Pierre Cerasi, de SINTEF Industry (Trondheim, Noruega) y Mohammad Masoudi de la Universidad de Oslo (Noruega).

% 
% \subsubsection{Visitantes recibidos}
% 
% \begin{itemize}
%  \item {\bf Luis Pugnaloni:} julio y diciembre, 2019. Investigador de la Universidad Nacional de La Pampa.
%  \item {\bf Lou Kondic:} julio 2019. Investigador del New Jersey Institute of Technology, Estados Unidos.
% \end{itemize}
% 
% \subsubsection{Visitas realizadas}
%  
%  \begin{itemize}
%   \item {\bf Manuel Carlevaro:} febrero 2019. Investigador visitante en el instituto de Química Física Rocasolano, Madrid, España.
%  \end{itemize}

\section{Actividades de gestión y evaluación}

Los miembros del GMG participan además en las siguientes actividades académicas y de gestión relacionadas con la investigación:

\begin{itemize}
    \item \textbf{Subsecretario de Ciencia y Tecnología UTN-FRLP:} Matías Fernández se desempeñó como Subsecretario de CyT durante 2023.

  \item \textbf{Consejo Asesor de Ciencia Tecnología y Postgrado UTN-FRLP:} Ramiro Irastorza fue miembro de la comisión durante 2023. 

\item \textbf{Comisión de Posgrado:} Ariel Meyra es miembro de esta Comisión que se conformó en 2023.
  
  \item \textbf{Referato de artículos para revistas internacionales}: durante 2023, C. Manuel Carlevaro fue revisor de artículos para las revistas \textit{Journal of Chemical Information and Modeling} (American Chemical Society), \textit{Journal of Vibration and Control} (SAGE Publications Inc.), \textit{Powder Technology} (Elsevier B. V.) y \textit{Lab on a Chip} (Royal Society of Chemistry). Marcos Madrid fue revisor de artículos para las revistas \textit{Granular Matter} (Springer), \textit{Processes} (Multidisciplinary Digital Publishing Institute) y \textit{Agriculture} (Multidisciplinary Digital Publishing Institute).
 
  \item \textbf{Actividades de gestión editorial:} C. Manuel Carlevaro fue Editor Invitado de ``Special Issue on Soft Matter Research in Latin America'' de \textit{Journal of Physics: Condensed Matter}, y Editor Asociado de \textit{Frontiers in Soft Matter}.
 
     \item \textbf{Evaluación de personal de CyT}: C. Manuel Carlevaro fue Especialista Externo en la evaluación de la Convocatoria Promoción CIC 2022 de CONICET. Marcos Madrid fue miembro evaluador de la Carrera del Personal de Apoyo de CONICET.

\item \textbf{Jurados de tesis}: C. Manuel Carlevaro fue miembro del jurado evaluador de la tesis doctoral de Rodrigo Caitano Barbosa da Silva (Facultad de Ciencias, Universidad de Navarra, España), y miembro del jurado de la tesina de grado de Julián María Gómez Paccapelo (Facultad de Ciencias Exactas y Naturales, Universidad Nacional de La Pampa).

\item \textbf{Jurados de concursos docentes}: C. Manuel Carlevaro fue miembro titular de jurados de diversos concursos para cargos de Profesor y Auxiliar Docente que se realizaron en el Departamento de Ingeniería Mecánica, de la Facultad Regional La Plata de la Universidad Tecnológica Nacional.

\item \textbf{Evaluación de proyectos de CyT}: C. Manuel Carlevaro fue evaluador de PICT de la Agencia Nacional de Promoción Científica y Tecnológica, en el área temática ``Tecnología Informática de las Comunicaciones y Electrónica''. R. Irastorza se desempeñó como Co-Coordinador de la comisión de Tecnología Informática, de las Comunicaciones y Electrónica en la evalución de Proyectos PICT de la Agencia Nacional de Promoción Científica y Tecnológica (ANPCYT); Ministerio de Ciencia, Tecnología e Innovación Productiva. 
 
\item \textbf{Asociación Física Argentina:} C. Manuel Carlevaro se desempeñó como Revisor de Cuentas Suplente. Marcos Madrid fue parte dewl comité ejecutivo de la división Materia Blanda.
 
\end{itemize}




\subsubsection{Tareas de divulgación}

Ramiro Irastorza participó como expositor en el Primer Workshop Argentino de Tomografía de Materiales, del 25 al 27 de octubre de 2023, organizado por Y-TEC. Título de charla: ``Tomografía por microondas: desarrollo de software y un prototipo experimental''.

\section{Registros y patentes}

No se realizaron registros ni patentes.



\chapter{Actividades en docencia}

\section{Docencia de grado}

Los integrantes del GMG se despempeñaron como docentes de las siguientes cátedras de la UTN-FRLP.

\begin{itemize}
 \item {\bf Mecánica de materiales granulares:} Matías Fernández.
 \item {\bf Mecánica de fluidos:} S. Mosca.
 \item {\bf Estimulación hidráulica de yacimientos no convencionales:} M. E. Fernández.
 \item {\bf Introducción a los elementos finitos:} R. Irastorza y A. Meyra.
 \item {\bf Fundamentos de Informática:} M. Madrid.
 \item {\bf Cálculo Avanzado:} L. Basiuk y Manuel Carlevaro.
\end{itemize}

%Además se participa como docente en otras casas de altos estudios.

%\begin{itemize}
 %\item {\bf Matemática C (Facultad Ing. UNLP):} M. A. Madrid.
%\end{itemize}


\section{Posgrado}

Los docentes del GMG son docentes en los siguientes cursos de postgrado.

\begin{itemize}
 \item {\bf Herramientas computacionales para científicos:} R. Irastorza. A. Meyra y C. M. Carlevaro.
 \item \textbf{Método de elementos finitos con software libre:} R. Irastorza y A. Meyra.
 \item \textbf{Estimulación hidráulica de yacimientos no convencionales:} M. Fernández.
 \item {\bf Análisis Estadístico utilizando R (Universidad Nacional Arturo Jauretche):} R. Irastorza.
\end{itemize}

\section{Otras actividades}

El banco de pruebas de descarga de silos montado en los laboratorios del GMG se utiliza para que estudiantes de las cátedras de grado realicen trabajos prácticos experimentales sobre flujo de materiales granulares y distribución de tensiones en un silo. 

M. Madrid se desempeñó como director de tesis de grado de Déborah Gozalez, y como codirector de tesis de grado de Ignacio Schulz, ambos alumnos de la Facultad de Ciencias Exactas de la Universidad Nacional de La Plata.

Santiago Mosca y Lucas Basiuk dictaron el curso ``Introducción a \LaTeX'' en el marco de la primera edición de la Semana de Postgrado (SEMPOST), con una extensión de 16 horas distribuidas en cuatro encuentros.


\chapter{Vinculación con el medio socioproductivo}

\section{Transferencia al medio socioproductivo}

Durante el año 2023 se desarrolló el proyecto de la Primera Convocatoria del Fondo de Innovación Tecnológica de Buenos Aires (FITBA), en colaboración con la fábrica de implementos agrícolas Heedba, de la localidad de 9 de Julio, cuyo propósito es el de desarrollar un sistema que optimice el consumo energético para la refrigeración de silos. 

\chapter{Informe sobre rendición general de cuentas}

Los valores presentados en la siguiente tabla son estimativos debido a que existen ingresos y erogaciones correspondientes a períodos diferentes del año 2022 dependiendo del inicio y cierre de los subsidios recibidos.

\vspace{1cm}
% Gastos capital: Inciso 4.3
% Gastos corrientes: Incisos 2 + 3
\begin{center}
\begin{tabular}{ l c c c }
 \toprule
 \textbf{Proyecto} & \textbf{Ingresos (\$)} & \multicolumn{2}{c} {\textbf{Egresos (\$)}} \\
            &           & \textbf{Capital} & \textbf{Corrientes} \\
\midrule
 UTN$^a$ & 550.000,00  & 450.000,00 & 100.000,00 \\
 MAUTILP0007746TC & 150.000,00 & 75.000,00 &  75.000,00 \\
 MAUTNLP0009875 & 540.000,00 & 270.000,00 & 270.000,00 \\
 MAECLP0009851TC & 540.000 & 270.000,00 & 270.000,00 \\
 PICT2020-SERIEA-00457 & 350.701,46 & 64.257,56 & 286.443,90  \\
 PICT-2020-SERIEA-I-GRF & 1.200.000,00  & 800.000,00 & 400.000,00  \\
 PICT-2021-I-A-00294 & 1.992.600,0 & 458.135,80 &  242.463,83 \\
 PIP 2021-2023 GI & 510.000,00 & 126.598,000 &  193.966,56 \\
 FITBA A64 & 7.850.000,00 & 7.000.000,00  &  850.000,00 \\
 \midrule
 \textbf{Total:} & XXX & XXX & XXX \\
 \bottomrule
\end{tabular}
\end{center}

\vspace{0.5cm}
$^a$ Financiamiento de la SCTyP de la UTN para grupos homologados.


\chapter{Programa de actividades 2023}

Las actividades planificadas para el año 2023 son:

\begin{itemize}
\item Iniciar el proyecto PID UTN MATCLP10087C, ``Estudio de propiedades dinámicas y estructurales de materiales granulares'' (1/4/2024 -- 31/3/2027). Analizar la dinámica de descarga de un silo en dos dimensiones sometido a vibraciones bi-armónicas por medio de simulaciones computacionales. 
\item Continuar con el desarrollo de los proyectos en ejecución:
    \begin{itemize}
\item PID: MAUTNLP0009875: Se continuará con las mediciones experimentales de presión en la base de los silos prototipo, y se construirán dos nuevos modelos complementarios para realizar experimentos con paredes de diferentes rugosidades. 
\item PID: MAECLP0009851TC: Continuación de trabajos con modelos de torso completo, ahora en lugar de evaluar ablaciones auriculares comenzaremos a trabajar en ablaciones pericárdicas en ventrículo. Adicionalmente, continuaremos con la simulación de la biomecánica de columna con discoplastía y comenzaremos con una línea de evaluación experimental y computacional de PMMA en modelos de vértebras con discoplastía.
\item PICT-2020-SERIEA-I-GRF-00457: Se realizarán mediciones de calibración con el prototipo desarrollado en medios canónicos (respuesta dieléctrica conocida) y se contrastarán con resultados analíticos y de simulación. Asimismo, se continuará con el desarrollo del \textit{software} de métodos directo e inverso.
\item PICT-2020-SERIEA-I-GRF-02611: Continuaremos con los trabajos de simulación sobre silos con paredes de diferentes rugosidades y medidas de presión tanto en simulaciones como en experimentos. La alumna de Lic. en Física presentará su trabajo final sobre los experimentos y simulaciones llevadas a cabo sobre carga y descarga de silos con diferentes protocolos de llenado.
\item PICT-2021-I-A-00294: Se finalizará el desarrollo del código de simulación de un amortiguador granular en dos dimensiones, cuya optimización geométrica se realizará por medio de un algoritmo genético. Se avanzará en el estudio experimental del efecto de la inclusión de obstáculos en un amortiguador granular. 
\item PIP 11220200100717CO: Se continuará trabajando con modelos teóricos sobre caudal de partículas por orificios con y sin forzado. Se estudiarán experimentos sobre optimización de flujo de partículas mono y polidispersas. 
\item FITBA A64: Se finalizará el proyecto realizando las mediciones del sistema optimizado en silos reales de la fábrica Heedba, en la ciudad de 9 de Julio.
\end{itemize}
\item  Redactar y publicar al menos siete trabajos en revistas internacionales con referato, producto de las investigaciones en las líneas de trabajo actualmente en desarrollo en el GMG.
\item  Participar en al menos tres congresos nacionales y uno internacional. 
\item  Progresar en el desarrollo de los planes de tesis doctorales en curso.
\item  Avanzar en la consolidación de las líneas de trabajo de los investigadores jóvenes. 
\item  Incorporar becarios estudiantes y graduados. 
\item  Continuar y consolidar las colaboraciones existentes con la empresa Y-TEC, el New Jersey Institute of Technology (USA), la Universidad de Navarra y el Instituto de Química Física Rocasolano (España). Mantener las colaboraciones activas con el Instituto de Física de Líquidos y Sistemas Biológicos, la Universidad Nacional de La Pampa, la Universidad de Buenos Aires y el Centro Atómico Bariloche.
\item  Dictar cursos de grado y posgrado.
\item  Incorporar a becarios del GMG a la Carrera del Docente Investigador UTN.
\item  Participar en las actividades que proponga la Secretaría de Ciencia y Tecnología de la Facultad Regional La Plata.
\end{itemize}


\end{document}
 
